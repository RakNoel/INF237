\documentclass[a4paper, 12pt]{article}

%Bookmarks
\usepackage[colorlinks=true,urlcolor=cyan,linkcolor=black,citecolor=red,bookmarksopen=true]{hyperref}
\usepackage{bookmark}

\usepackage[utf8]{inputenc}
\usepackage{amsmath}
\usepackage{pgf,tikz}
\usepackage{mathrsfs}
\usepackage{listings}
\usetikzlibrary{arrows}
\usepackage{amssymb}
\usepackage{url}
\usepackage{epigraph}

%Margins
\usepackage[margin=1.0in]{geometry}

%Images%
\usepackage{graphicx}
\usepackage{float}

%Citations
\usepackage[round]{natbib}
\bibliographystyle{plainnat}

%Code
\usepackage{listings}
\usepackage{color}

\definecolor{dkgreen}{rgb}{0,0.6,0}
\definecolor{gray}{rgb}{0.5,0.5,0.5}
\definecolor{mauve}{rgb}{0.58,0,0.82}

\lstset{frame=tb,
  language=Java,
  aboveskip=3mm,
  belowskip=3mm,
  showstringspaces=false,
  columns=flexible,
  basicstyle={\small\ttfamily},
  numbers=none,
  numberstyle=\tiny\color{gray},
  keywordstyle=\color{blue},
  commentstyle=\color{dkgreen},
  stringstyle=\color{mauve},
  breaklines=true,
  breakatwhitespace=true,
  tabsize=3
}

\newcommand{\mysection}[1]{\section*{#1} \addcontentsline{toc}{section}{#1}}
\newcommand{\mysubsection}[1]{\subsection*{#1} \addcontentsline{toc}{subsection}{#1}}
\newcommand{\mysubsubsection}[1]{\subsubsection*{#1} \addcontentsline{toc}{subsubsection}{#1}}

\newcommand{\myFigure}[3]{\begin{figure}[h!]\centering\includegraphics[scale=#1]{figures/#2}\caption{#3}\end{figure}}

\newcommand{\Problem}[4]{\subsection*{#1} \addcontentsline{toc}{subsection}{#1}
\centerline{\textit{Programing language: #2} \quad \textit{Difficulty: #3} \quad \textit{Est: #4 hrs.}}}

\newcommand{\mycitation}[1]{[\citet{#1}]}

\begin{document}

    % % % % % % % % % % % % % % % % %
    %
    %	FRONT PAGE
    %
    \begin{titlepage}
    \newcommand{\HRule}{\rule{\linewidth}{0.5mm}}
    \center

    \textsc{\LARGE University of Bergen}\\[1.5cm] % Name of your university/college
    \textsc{\Large INF237}\\[0.5cm] % Major heading such as course name
    \textsc{\large Algorithm Engineering}\\[0.5cm] % Minor heading such as course title

    \HRule \\[0.4cm]
    { \huge \bfseries Personal notes from the problem solving}\\[0.4cm] % Title of your document
    \HRule \\[1.5cm]

    \Large \emph{Author:}\\
    Oskar Leirvåg (OLE006)
    \\[2cm] % Your name

    \centerline{\includegraphics[scale=0.15]{figures/canvas}} % Include a department/university logo - this will require the graphicx package

    {\large \today}\\[3cm] % Date, change the \today to a set date if you want to be precise

    \vfill
\end{titlepage}


    % % % % % % % % % % % % % % % % %
    %
    %	TABLE OF CONTENTS
    %
    \pdfbookmark{\contentsname}{toc}
    \tableofcontents
    \newpage

    % % % % % % % % % % % % % % % % %
    %
    %	Abstract
    %
    \mysection{Abstract}
    This document is meant to give an insight in how the algorithms were designed. The intention is to give a deeper 
    explanation to the thought process in the hopes of helping others understand while also helping myself to remember
    how it was done. It is important to mention that this course does require an oral test at the end of the semester
    and this text will help me in the preparation for this.
    \\
    This course uses Kattis as a tool to publish the weekly challenges, and also test submissions. The algorithms that are submidted are
    thurougly tested over many hidden tests to affirm that they do fulfill all requirements. In order to pass this course a given
    number of problems must be completed each week, and they usually hold a theme in common.

    \newpage
    
    % % % % % % % % % % % % % % % % %
    %
    %	Assignment 1
    %
	\mysection{Assignment 1: Ad-hoc}	
    The first week was obviusly only a few simple problems as to get to know the Kattis tool. I used this time to experiment with some different programming languages like Java, C++
    and Haskell. 

    \Problem{Taks one: Oddities}{Java}{Beginner}{0.5}
    This problem was really straight foreward. Given the input of any random number \textit{n} of a 
    given size, write if the given number is \textit{odd} or \textit{even}. So I guess there is
    little to say about the logic here. Could possibly have been solved even faster in Haskell than java.

    \Problem{Taks two: Different Problem}{Haskell}{Beginner}{0.5}
    This problem simply gives you two numbers \textit{a} and \textit{b} and wants the output of the \textbf{absolute distance}. This is solved
    quite simply in Haskell with the recursive function
    \lstset{language=Haskell}
    \begin{lstlisting}
        solve :: [Integer] -> [Integer]
        solve [] = []
        solve (a:b:xs) = abs(a - b):(solve xs)
    \end{lstlisting}

    \Problem{Taks three: Guess the number}{C++}{Easy}{2}
    This is a simple simulation of the old guessing game. The computer decides a number $1 \leq n \leq 1000$, and your algorithm should be able to guess the correct
    number within 10 questions given an honest responce of higher/lower/correct. This has to be done by a binary search.

    \Problem{Taks four: Turtle master}{C++}{Easy}{3}
    In this problem we are given a game-board with a robot, a goal and some obsticles. Then we are given inputs (commands) for the robot to follow, and wee
    shall simply simulate the result. Then if all criteria is met we should return either \textit{Diamond!} if correct, or \textit{Bug!} if any error/illegal move occurs.
    \\
    The short solution to this problem was simply a full on object oriented approach. I simply made a Position object and read from the map if any new positions was valid. Now
    finally if the robot was standing on the goal in the end, print \textit{"Diamond!"}. Now this took a lot of extra time due to me not knowing C++ and how to use pointers.

    % % % % % % % % % % % % % % % % %
    %
    %	ADD REFERANCES
    %
    %\bibliography{citation-db}
    %\addcontentsline{toc}{section}{References}

\end{document}